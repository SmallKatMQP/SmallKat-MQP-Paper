\graphicspath{ {./figures/} }
\section{Background}
\label{chap:background}
\subsection{Introduction}
%need to fix
    As robotic systems become more complicated and the amount of computing power we have available to us increases, we have slowly seen more and more complex robot systems rise up. While many of these robots still use traditional wheels as a form of locomotion the legged robots are slowly becoming more popular. When it comes to traversing complex terrains climbing up obstacles such as stairs and rocky mountain sides there are no better systems than legged ones. As a result these robots are often designed for reconnaissance or search and rescue. Additionally the military has tested many of these robots as robotic mules. Using the robots to  carry heavy items or people through the rough battlefield terrain. Quadrupeds are complex systems to create but they are getting better and better and we may soon start to see quadrupeds being used in more and more situations where typical robotic systems just wont do.
\subsection{Small(Kat) Beginnings}
 The SmallKat project started in October 2017 as an Independent Study Project (ISP) with Worcester Polytechnic Institute (WPI) Professor Ciaraldi. The original intention of the project was to create the smallest possible walking quadruped we could in a single term. Using a digital 9 gram servo and a custom circuit board as the back bone of the project we were able to create a 16 degree of freedom system that very much resembled a feline. The first version barely walked and while ultimately the results of this project were quite successful, it left us wanting a much better solution. During the summer of 2018 we continued the project creating a slightly larger more advanced system. This version dubbed SmallKat V2 was significantly more successful than its predecessor. It was more reliably designed and used much more reliable servo motors and several sensors including IMUs, pressure sensitive feet and current sensing. Additionally a reliable static walking gait was created and tested allowing it become WPI's most successful quadrupedal robotic system. Two versions of SmallKat V2 were built, aptly named Crimson and Gray they have been demoed at many of WPI's undergraduate robotics demonstrations. 
     \begin{figure}[H]
        \centering
        \includegraphics[width=120mm]{figures/V1andV2.jpg}
        \caption{SmallKat Versions 1, 2 and 2.1 during an undergraduate robotics lab demonstration}
        \label{fig:my_label}
    \end{figure}
\subsubsection{SmallKat V1 \& V2}
    \paragraph{Electrical}
    Developing V1 \& V2 of the SmallKat project revealed a lot of nuances when it came to the electrical system. These stemmed from the lack of any form of fed back due to the limited capabilities of using PWM and hobby servos. On V1 we had a very simplistic motherboard, just enough to get the system working. The purpose of the board was to breakout the 22 servo ports available on the Teensy 3.5 micro-controller as well as regulate the 7.4-8.4v input from the battery to 5v for the Teensy and the raspberry pi zero. This board made connecting all 16 Servos much easier and cleaner. When developing the motherboard for V2 a lot more time was taken to consider the possibilities of what could be achieved from the motherboard. On version 2, USB communication between the Raspberry Pi 3 B+ and the Teensy 3.5 was used to increase communication speed. IMUs were implemented at the front and rear of the robot along with current sensors for each of the 16 motor. Integration of these sensors were our attempt to combat the lack of feedback from our system. While we still had no direct positional or velocity feed back from the motors we could still attempt to implement a rough closed loop control using IMU data or determine the torque on the motors using the current sensing. Additional pressure sensors were added in the feet to detect when the robot was in contact with the ground. Due to time constraints these sensors have yet to be fully implemented into the control system of SmallKat V2.
    \paragraph{Computer Science}
    \subparagraph{Embedded}
    The Teensy 3.5 was programmed and developed through the Arduino environment. This was done in order to reduce the development time. However, this resulted in a number of issues arising. Because the target market for the Arduino environment is primarily beginners and hobbyists, there exist a great number of unnecessary checks and even more unnecessary layers of abstraction. This inadvertently limited the speed we could run the robot at. Many issues with using 16 PWM channels also arose at different points in the project as the platform is only designed to handle 12 by default, this meant a number of internal libraries had to be modified before the motors would move as expected. 
    \subparagraph{Kinematics and Dynamics}
    For the higher level control of the robot a program called Bowler Studio was used. This program was developed by WPI graduate and Undergraduate Robotics Lab Manger Kevin Harrington. All the kinematic, dynamic and trajectory planning calculations were done using bowler and then sent to the robots using the HID protocol. Due to this program being in an early Beta stage there were a number of bugs that slowed the progress of the project. However having the developer of the program as an advisor to the project many of these were resolved quickly and/or could be be worked around relatively easily. This solution proved to be quite easy to use and allowed us to quickly develop a basic static walking gait. 
    % Need to cite Bowler Studio in this paragraph
    \paragraph{Mechanical}
    the mechanical aspect of the SmallKat system is centered around the servo for the system. For version 1 we used a small 9 gram digital servo. These provided a high amount of torque for their size. However they had a tendency to seize up if they were rotated not under their own power. Additionally the plastic used to house them was brittle and the small flanges used to hold them in place would often break. Other design flaws of the V1 system included hard to reach bolts and cheap screws used to connect each link to the servo horn. These screws would often strip when removed which happened often because the links would have to be adjusted. Version 2 fixed many of these problems. The servos were held inside each link with a cover instead of bolting through the flanges. Additionally a thin straight lever arm was used to connect each joint to the servo. This was pressed on after the link was already installed allowing for it to be easily adjusted. From version 1 to 2 the overall design and look didn't change. The robot had 4 legs wit 3 degrees of freedom and were situated underneath the body along with movable head and tail. While very similar in design to other quadrupedal systems The cat like features brought a smile to many faces.
    \subsubsection{SmallKat V3}
    The SmallKat MQP is the next evolution of the SmallKat quadruped series. Based on our prior work we are looking to create a new quadruped with more advanced systems than the previous two versions to allow  us to experiment with more complicated walking gaits and control systems.

\subsection{Other Robots}
\subsubsection{Notable Quadrupedal Robots}
Boston Dynamics is currently one of the leaders in quadrupedal and bipedal systems. Their robots are the source of much inspiration for people attempting to create quadruped systems. They have created a variety of quadrupeds each with different capabilities
\paragraph{Boston Dynamics’ Spot Mini}
SpotMini is a quadruped system designed and built by Boston Dynamics. It is a small agile system intended to be used in a variety of applications such as the home, office or outdoors. It stands at .83 meters tall and weighs 25kg. SpotMini is all-electric and can go for about 90 minutes on a single charge, depending on what it is doing.     
     \begin{figure}[H]
        \centering
        \includegraphics[width=120mm]{figures/SpotMini.jpg}
        \caption{Boston Dynamics Spot Mini}
        \label{fig:my_label}
    \end{figure}
SpotMini uses 4, 3 DOF legs to move itself around. It sports a 5kg, 5 DOF arm that it can use to pick things up and open doors. The sensor suite on board SpotMini and includes stereo cameras, depth cameras, an IMU, and position/force sensors in the limbs. These sensors help with navigation and mobile manipulation. SpotMini's dynamic control is unparalleled by any other quadruped system. The robot's movement is fluid and almost life like and it is able to navigate complex terrain and stairs. SpotMini can often be seen on YouTube were Boston Dynamics posts the robots new abilities and accomplishments. They plan on being able to commercial produce spot mini as a research platforms and for use in offices and homes.
\begin{figure}[H]
        \centering
        \includegraphics[width=120mm]{figures/SpotMiniwithArm.jpg}
        \caption{Spot Mini with Arm}
        \label{fig:my_label}
    \end{figure}

    \paragraph{MITs Cheetah}

\subsection{Boston Dynamics}
    During the course of the project we reached out to Boston dynamics and arranges a conference call between our team and a group of their engineers from the spot team. We chose to reach out tho them as they are the current leader in the quadrupedal robotics field and would therefore have the most insight into the intricacies and problems that arise in the development of such a platform. We entered the meeting with a series of questions and subordinate related questions. This list comprised of:
    \begin{itemize}
        \item Why was the decision mate to pursue 3DOF legs instead of 4DOF as most quadrupedal animals have. 
        \begin{itemize}
            \item Why did you choose to abstain from a head and tail for balance and dynamics
            \item What are some of the different gaits you have developed and why did you choose them
            \item How have you obtained such a smooth waling gait.
        \end{itemize}
        \item How are your trajectories and kinematics calculated pragmatically
            \begin{itemize}
                \item Is a real time physics engine used
                \item Are jacobians calculated at run time
            \end{itemize}
        \item What kind of sensors are used on Spot mini
            \begin{itemize}
                \item How is each sensor integrated and used for corrections
                \item recommendations on where to locate each kind of sensor
            \end{itemize}
        \item How importance is torque control in dynamic gaits
        \item What was the minimum control loop speed you found was necessary for smooth operation
    \end{itemize}
These questions stemmed a very informative conversation between ourselves and the team of engineers which comprised of 2 Electrical engineers and 1 Mechanical engineer, sadly no controls engineers were able to participate however those in attendance were very knowledgeable and were able to answer the majority of our controls related questions. From this conversation we gained a great deal of insight into the tough process that went into developing Spot mini as well as many previous robots in the Spot and big dog series.  

We started off by asking about the decisions that led to them choosing to only use 3 DOF legs instead of using 4 DOF legs as most quadrupedal animals in nature do. To this we were promptly responded when possible keep the mechanism and calculations as simple as possible, they had tested 4DOF legs on previous robots and found they generally avoided using them and that everything they needed to do could be achieved using 3 DOF, In addition using only 3 DOF also cut down on the wiring, and weight in the legs and therefore lowered the inertial mass of the robot making it over all easier to control. They continued on to speak to the benefits of adding a joint near the center of the body in order to act as a spine which would be much more useful when it cam to highly dynamic gaits and motions allowing for the body to fold more and allow for a smoother transference of momentum.

This followed into the choice of force sensing, For the project we chose to use an array of barometric pressure sensors encapsulated in polyurethane to calculate a pressure vector. The engineers at Boston dynamics had actually tested a similar kind of sensor and had a great deal of issues using them of their robots as the repetitive motion in walking would either damage of change the calibration of the sensor. In combination with this the temperature of the environment had a great effect of the polyurethanes ability to transmit the applied force to the sensors. This combined with the damage that would get caused to the polyurethane during outdoor testing led them to decide not to continue using this method of sensing. Instead they choose to sense force at the actuator. Their recommendation was to use current to propagate torque at the motor as they used a combination of this and a custom force sensor that they could not speak too much to due to it being part of their "robot secret sauce". 

Following this we asked about the development of gaits and their choice of a trot gait in which the legs are moves in a FR, BL, BR, FL sequence instead of a more natural pace gait in which the legs are moved in a FR, BR, FL, BL sequence. Their response was different to what we had expected, in end Spot is able to perform a pace gait however it is less stable than the trot gait. The trot gait was the first successful gait the team was able to develop when working on the original spot project and therefore has had the most development put into it. Further gait development came with difficulties, the development of a bounding or galloping gait was severely hindered by the lack of an effective spine, without this the body would form an oscillating system making the robot unstable. However when mentioned the concept of a continuum spine which would result in the greatest mobility they greatly recommended against it, this was due to the uncertainty and inability to guarantee the position of the system for a given input. For all dynamic motions such as recovering from a push or jerk, they rely heavily on the IMU which in a ridged body robot they recommend placing it directly over the center of the robot however for a split body or one using a spine they recommended placing one on each segment in order to capture the motion of each independent system. 

Leading from this to kinematics and trajectory planning allowed us to learn that they do use both a simulation and a real time physics engine as well as calculation all necessary jacobians at run time. This led us to asking details of their control feedback and they recommended getting a high level controller running at \~1 kHz for a smooth trajectory and a low level controller as fast as possible in order to have a stable PID loop and remove all gittering and jerkiness from the system.  


