\renewcommand{\abstractname}{Executive Summary}
\begin{abstract}
SmallKat will be designed to fill a void in the research and development of multipedal robotic systems. Currently, quadruped dynamics and gaits are developed using simulation of robotic platforms costing upwards of \$100,000. This becomes very impractical for small companies, universities, and hobbyists alike due to the high fixed cost. SmallKat is intended to provide a quadrupedal platform to help research and design new gaits, test sensors and motors, and teach up-and-coming engineering students. Where-as the current options only allow for a limited number of robots due to cost, size, complexity, and safety constraints, SmallKat will allow for organizations to have multiple development platforms running at the same time, allowing for comparisons of components and walking gaits. At the same time, SmallKat is attainable by private hobbyists who are interested in learning more and developing more complex gaits for quadrupeds. This will exponentially increase the research on quadrupedal walking systems industry and academia alike. SmallKat will be designed to be an adaptable quadruped. Therefore, the system must be highly modular and easily repairable. To allow for a multitude of different gaits, SmallKat will have four 4 degrees-of-freedom (DoF) legs instead of the normal 3 on other platforms. Each joint will be controlled by a powerful servo motor, with possible spring assistance to reduce the current draw of the motor. These motors will be a modified version of standard hobby servos and will be combined with custom sensors to allow for various forms of feedback including position, velocity, torque feedback from each motor, Triangulated pressure vectors for contact point detection on each foot and body linear velocity, angular acceleration and gyro data. To control the SmallKat we intend on creating and easily usable software platform that allows for a smooth development and testing of the SmallKat system.
\end{abstract}
