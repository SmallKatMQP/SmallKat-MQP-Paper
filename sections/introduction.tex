\graphicspath{ {./figures/} }
\section{Introduction}
\label{chap:Introduction}
%need to fix
% IQP Paper Guidelines 
% https://www.wpi.edu/sites/default/files/IQP_Writing_Guidelines.pdf

% QRP Paper
% https://web.wpi.edu/Pubs/E-project/Available/E-project-042518-003336/unrestricted/QRP_FinalReport.pdf
%Robodog
%https://web.wpi.edu/Pubs/E-project/Available/E-project-042717-152039/unrestricted/MQP_RoboDog_Final_Report.pdf

% The Introduction chapter of an IQP report serves to introduce the reader to your general area of
% inquiry, clarify the specific challenge you will address, and explain why the work is worthwhile. The
% Introduction is usually no more than two pages and follows a set of “rhetorical moves” that you may
% have seen when reading experimental or research papers. The five moves, in order, are to:



% Quadruped robotics. terrestrial robotics or robots that use other means of transportation besides wheels or tracks 

%  


% 

% Lack of affordable, complex quadruped systems that allow students or other robotic enthusiasts to explore the realm of quadruped systems and different walking gait styles.

 

% Smallkat was the first iteration of this project. created a small scale version of the quadruped system in this project. provided background knowledge of quadrupedal systems. Not many commercial products available for a low price to learn how to create a quadruped system let alone university classes that have access to physical quadruped systems. Many are done in simulation or theoretical. Other affordable systems on the market do not provide the level of complexity necessary to create a useful product.


% Need for a complex, affordable system that can be used to explore the nature of quadrupeds or walking gaits

% create a 4 dof system with a continuum tail could be used as a basis for further research. additional research could be done in the exploration of artificial pets with the ability to provide comfort to mentally disabled or elderly with out the need to actually take care of a real animal.


% As robotic systems become more complicated and the amount of computing power we have available to us increases, we have slowly seen more and more complex robot systems rise up. While many of these robots still use traditional wheels as a form of locomotion the legged robots are slowly becoming more popular. When it comes to traversing complex terrains climbing up obstacles such as stairs and rocky mountain sides there are no better systems than legged ones. As a result these robots are often designed for reconnaissance or search and rescue. Additionally the military has tested many of these robots as robotic mules. Using the robots to  carry heavy items or people through the rough battlefield terrain. Quadrupeds are complex systems to create but they are getting better and better and we may soon start to see quadrupeds being used in more and more situations where typical robotic systems just wont do.

    
%The motivation behind the development of an affordable yet capable quadrupedal robotic platform was to advance the research and development of multipedal systems.
% 1. Establish the general topic, problem, or field that your project addresses
% 2. Introduce the specific problem or issue;
One of the hardest tasks for robotics is maneuverability over rough terrain. Legged robots have the unique capability of having a high ground clearance while still being more stable than traditional wheeled robots. Quadrupeds take that one step further by making the platform more stable than its bipedal counterpart, like seen in Figure %Insert figure
The design and research of quadruped systems are inspired and motivated by the limitations of wheeled and tracked robotic systems. These systems favor flat terrain and struggle with extremely rough, uneven or rocky terrain, while quadrupedal systems excel in these environments. When it comes to traversing complex terrains climbing up obstacles such as stairs and rocky mountain sides there are no better systems than legged ones. Quadruped systems have huge potential because of their ability to navigate these nonuniform surfaces, and are ideal for exploration or providing assistance in the home. Using quadrupeds to explore locations wheeled robots can't is not just important here on earth but would also open a world of opportunities on other planets or moons in our strive for space exploration. In the home, quadrupeds can be used to navigate stairs or obstacles on the ground. They would allow us to create robotic assistants without having to restructure our homes or workplaces to accommodate a wheeled robot. Additionally they have many applications in the military as well. These kinds of robots are already being designed for reconnaissance or search and rescue and the United States military has contracted several of these robots to be used as robotic mules that can carry heavy items or people through the rough battlefield terrain. Quadrupeds are complex systems to create but they are getting better and better and we may soon start to see quadrupeds being used in more and more situations where typical robotic systems just wont do. However, even though the usage of quadruped systems has increased, there is still a lack of affordable yet capable quadruped systems that allow students or other robotic enthusiasts to explore the realm of quadrupedal systems. This makes it very difficult for the industry to expand because currently the only way to learn, experiment, develop or create anything useful is if you are able to get your hands on a multi-million dollar plate form such as those used by several high profile companies or organizations like DARPA or Boston Dynamics. 

% 3. Define the scope of problem or issue by summarizing previous work, what is already known,and/or what has already been done;
In previous years there have been many MQPs who have attempted to make a working quadruped and none have succeeded. Before beginning this project we looked into the work done by the QRP MQP , the RoboDog MQP and the HydroDog MQP. Through analyzing their research and their project, we discovered a fatal flaw of every project was the use of 2 DOF legs and no other means of balancing. This lack of a 3rd or even 4th DOF limited the robots ability to stabilize itself or even turn as the leg was unable to be swung in or out to regain balance. This severely limited the capabilities of the robot and would therefore be the first thing we would address when designing this new platform.  This combined with the lack of adequate torque analysis gave us a deep understanding into the downfalls of the projects. 
 
% 4. Create a research space by identifying a gap in previous work or opportunities for extension or new work (motivating but not quite declaring your particular project by establishing specific need)

% 5. Establish your project by explicitly stating its goal, how it was achieved (objectives), Justifying it, and suggesting its implications and possible impact.

% create a 4 dof system with a continuum tail could be used as a basis for further research. additional research could be done in the exploration of artificial pets with the ability to provide comfort to mentally disabled or elderly with out the need to actually take care of a real animal.

SmallKat MQP will develop a small robotic quadruped for research and development purposes. It will be designed with 4 degree-of-freedom (DoF) legs, a continuum tail, and 3D printed body and construction. This will all be powered by custom servo motor controllers and 9DoF IMU sensors, all connected to a custom embedded SPI daisy chain protocol running on a custom microprocessor. The higher level controller will be running on a single-board computer for added performance when running kinematics and dynamics algorithms. To prove its capability, a basic walking gait will be developed. Eventually, SmallKat will be used by academic institutions, corporations, or hobbyists that are interested in further developing multipedal robotics platforms. Since SmallKat will be open sourced, anyone will be able to modify both the software and hardware to their need. This opens up possible future research opportunities, including but not limited to new gaits for quadrupedal platforms, continuum spines and necks for more mobility, and psychological research in the effect of robotic pets for mentally disabled citizens. An inexpensive quadrupedal platform will decrease the barrier of entry in robotics and allow for more innovation in the field.

